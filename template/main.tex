\documentclass[a4paper]{article}

\usepackage{amsmath}
\usepackage{amssymb}
\usepackage{cmap}
\usepackage{geometry}
\usepackage{hyperref}
\usepackage{indentfirst}
\usepackage{xeCJK}
\usepackage{minted}

\geometry{margin=1in}

\setmainfont{Microsoft YaHei UI}
\setCJKmainfont[BoldFont={Microsoft YaHei UI}]{Microsoft YaHei UI}
\setCJKmonofont{Microsoft YaHei UI}

\newcommand{\cppcode}[1]{
    \inputminted[mathescape]{cpp}{source/#1}
}

\newcommand{\javacode}[1]{
    \inputminted[mathescape]{java}{source/#1}
}

\title{代码库}
\author{上海交通大学}
\date{\today}

\begin{document}

\maketitle

\tableofcontents

\clearpage

\section{数论}

\subsection{快速求逆元}

返回结果:$$x^{-1}(mod)$$
\indent 使用条件:$x \in [0, mod)$并且$x$与$mod$互质。

\cppcode{number-theory/inverse.cpp}

\subsection{扩展欧几里德算法}

返回结果:$$ax+by=gcd(a,b)$$
\indent 时间复杂度:$O(nlogn)$

\cppcode{number-theory/extended-euclid.cpp}

\subsection{中国剩余定理}

返回结果:$$x \equiv r_i (mod \ p_i) \ (0 \leq i < n)$$
\indent 使用条件:$p_i$无需两两互质\\
\indent 时间复杂度:$O(nlogn)$

\cppcode{number-theory/chinese-remainder-theorem.cpp}

\subsection{Miller Rabin 素数测试}

\cppcode{number-theory/miller-rabin.cpp}

\subsection{Pollard Rho 大数分解}

时间复杂度:$O(n^{1/4})$

\cppcode{number-theory/pollard-rho.cpp}

\subsection{快速数论变换}

\subsection{原根}

\subsection{离散对数}

\subsection{离散平方根}

\subsection{佩尔方程求解}

\subsection{牛顿迭代法}

\subsection{直线下整点个数}

返回结果:$$\sum_{0 \leq i < n} \lfloor \frac{a + b \cdot i}{m} \rfloor$$
\indent 使用条件:$n, m > 0$,$a, b \geq 0$\\
\indent 时间复杂度:$O(n log n)$

\cppcode{number-theory/lattice-count.cpp}

\section{数值}

\subsection{高斯消元}

\subsection{快速傅立叶变换}

返回结果:$$c_i=\sum_{0 \leq j \leq i} a_j \cdot b_{i-j} \ (0 \leq i < n)$$
\indent 时间复杂度:$O(n log n)$

\cppcode{numerical-algorithm/fast-fourier-transform.cpp}

\subsection{单纯形法求解线性规划}

返回结果:$$max\{c_{1 \times m} \cdot x_{m \times 1} \ | \ x_{m \times 1} \geq 0_{m \times 1}, a_{n \times m} \cdot x_{m \times 1} \leq b_{n \times 1}\}$$

\cppcode{numerical-algorithm/linear-programming-simplex.cpp}

\subsection{自适应辛普森}

\cppcode{numerical-algorithm/adaptive-simpson.cpp}

\subsection{多项式方程求解}

\subsection{最小二乘法}

\section{数据结构}

\subsection{平衡的二叉查找树}

\subsubsection{Treap}

%\cppcode{data-structure/treap.cpp}

\subsubsection{Splay}

\subsection{坚固的数据结构}

\subsubsection{坚固的线段树}

\cppcode{data-structure/persistent-segment-tree.cpp}

\subsubsection{坚固的平衡树}

%\cppcode{data-structure/persistent-treap.cpp}

\subsubsection{坚固的字符串}

\subsubsection{坚固的左偏树}

\subsection{树上的魔术师}

\subsubsection{轻重树链剖分}

%\cppcode{data-structure/heavy-light-decomposition.cpp}

\subsubsection{Link Cut Tree}

\subsubsection{AAA Tree}

\subsection{k-d树}

\section{图论}

\subsection{强连通分量}

%\cppcode{graph-theory/strongly-connected-components.cpp}

\subsection{双连通分量}

\subsubsection{点双连通分量}

\subsubsection{边双连通分量}

\subsection{2-SAT问题}

%\cppcode{graph-theory/two-satisfiability.cpp}

\subsection{二分图最大匹配}

\subsubsection{Hungary算法}

时间复杂度:$O(V \cdot E)$

\cppcode{graph-theory/maximum-matching-hungary.cpp}

\subsubsection{Hopcroft Karp算法}

时间复杂度:$O(\sqrt{V} \cdot E)$

\cppcode{graph-theory/maximum-matching-hopcroft-karp.cpp}

\subsection{二分图最大权匹配}

\subsubsection{KM算法}

时间复杂度:$O(V^3)$

\cppcode{graph-theory/maximum-weight-matching.cpp}

\subsubsection{扩展KM算法}

\subsection{最大流}

时间复杂度:$O(V^2 \cdot E)$

\cppcode{graph-theory/maximum-flow.cpp}

\subsection{最小费用最大流}

\subsubsection{稀疏图}

时间复杂度:$O(V \cdot E^2)$

\cppcode{graph-theory/minimum-cost-flow-spfa.cpp}

\subsubsection{稠密图}

时间复杂度:$O(V \cdot E^2)$

\cppcode{graph-theory/minimum-cost-flow-zkw.cpp}

\subsection{一般图最大匹配}

时间复杂度:$O(V^3)$

\cppcode{graph-theory/maximum-matching-blossom.cpp}

\subsection{无向图全局最小割}

时间复杂度:$O(V^3)$\\
\indent 注意事项:处理重边时,应该对边权累加

\cppcode{graph-theory/minimum-cut-stoer-wagner.cpp}

\subsection{最小树形图}

\subsection{有根树的同构}

时间复杂度:$O(V log V)$

\cppcode{graph-theory/rooted-tree-isomorphism.cpp}

\subsection{度限制生成树}

\subsection{弦图相关}

\subsubsection{弦图的判定}

\subsubsection{弦图的团数}

\subsection{哈密尔顿回路(ORE性质的图)}

ORE性质:$$\forall x,y \in V \wedge (x,y) \notin E \ \ s.t. \ \ deg_x+deg_y \geq n$$
\indent 返回结果:从顶点$1$出发的一个哈密尔顿回路\\
\indent 使用条件:$n \geq 3$

\cppcode{graph-theory/hamiltonian-circuit-ore.cpp}

\section{字符串}

\subsection{模式匹配}

\subsubsection{KMP算法}

\cppcode{string-manipulation/knuth-morris-pratt.cpp}

\subsubsection{扩展KMP算法}

\subsubsection{AC自动机}

%\cppcode{string-manipulation/aho-corasick-automation.cpp}

\subsection{后缀三姐妹}

\subsubsection{后缀数组}

%\cppcode{string-manipulation/suffix-array.cpp}

\subsubsection{后缀自动机}

\cppcode{string-manipulation/suffix-automation.cpp}

\subsection{回文三兄弟}

\subsubsection{Manacher算法}

\cppcode{string-manipulation/manacher.cpp}

\subsubsection{回文树}

\cppcode{string-manipulation/palindrome-tree.cpp}

\subsection{循环串最小表示}

\cppcode{string-manipulation/minimum-circular-representation.cpp}

\section{计算几何}

\subsection{二维基础}

\subsubsection{点类}

\subsubsection{凸包}

\cppcode{computational-geometry/convex-hull.cpp}

\subsubsection{半平面交}

\subsection{三维基础}

\subsubsection{点类}

\subsubsection{凸包}

\subsubsection{绕轴旋转}

\subsection{多边形}

\subsubsection{判断点在多边形内部}

\subsubsection{旋转卡壳}

\subsubsection{动态凸包}

\subsubsection{点到凸包的切线}

\subsubsection{直线与凸包的交点}

\subsubsection{凸多边形的交集}

\subsubsection{凸多边形内的最大圆}

\subsection{圆}

\subsubsection{圆类}

\subsubsection{圆的交集}

\subsubsection{最小覆盖圆}

\subsubsection{最小覆盖球}

\subsubsection{判断圆存在交集}

\subsubsection{圆与多边形的交集}

\subsection{三角形}

\subsubsection{三角形的内心}

\subsubsection{三角形的外心}

\subsubsection{三角形的垂心}

\subsection{黑暗科技}

\subsubsection{平面图形的转动惯量}

\subsubsection{平面区域处理}

\subsubsection{Vonoroi图}

\section{其他}

\subsection{某年某月某日是星期几}

\cppcode{miscellany/what-day-is-today.cpp}

\subsection{动态规划}

\subsection{搜索}

\subsubsection{Dancing Links X}

\section{Java}

\subsection{基础模板}

\javacode{template.java}

\subsection{BigInteger}

\subsection{BigDecimal}

\section{数学}

\subsection{常用积分表}

\subsection{常用数学公式}

\subsubsection{求和公式}

\begin{enumerate}
\item
  $\sum_{k=1}^{n}(2k-1)^2 = \frac{n(4n^2-1)}{3}	$
\item
  $\sum_{k=1}^{n}k^3 = [\frac{n(n+1)}{2}]^2	$
\item
  $\sum_{k=1}^{n}(2k-1)^3 = n^2(2n^2-1)	$
\item
  $\sum_{k=1}^{n}k^4 = \frac{n(n+1)(2n+1)(3n^2+3n-1)}{30}  $
\item
  $\sum_{k=1}^{n}k^5 = \frac{n^2(n+1)^2(2n^2+2n-1)}{12}	$
\item
  $\sum_{k=1}^{n}k(k+1) = \frac{n(n+1)(n+2)}{3}	$
\item
  $\sum_{k=1}^{n}k(k+1)(k+2) = \frac{n(n+1)(n+2)(n+3)}{4} $
\item
  $\sum_{k=1}^{n}k(k+1)(k+2)(k+3) = \frac{n(n+1)(n+2)(n+3)(n+4)}{5} $
\end{enumerate}

\subsubsection{斐波那契数列}

\begin{enumerate}
	\item $fib_0=0, fib_1=1, fib_n=fib_{n-1}+fib_{n-2}$
	\item $fib_{n+2} \cdot fib_n-fib_{n+1}^2=(-1)^{n+1}$
	\item $fib_{-n}=(-1)^{n-1}fib_n$
	\item $fib_{n+k}=fib_k \cdot fib_{n+1}+fib_{k-1} \cdot fib_n$
	\item $gcd(fib_m, fib_n)=fib_{gcd(m, n)}$
	\item $fib_m|fib_n^2\Leftrightarrow nfib_n|m$
\end{enumerate}

\subsubsection{错排公式}

\begin{enumerate}
	\item $D_n = (n-1)(D_{n-2}-D_{n-1})$
	\item $D_n = n! \cdot (1-\frac{1}{1!}+\frac{1}{2!}-\frac{1}{3!}+\ldots+\frac{(-1)^n}{n!})$
\end{enumerate}

\subsection{平面几何公式}

\subsubsection{三角形}

\begin{enumerate}
	\item 半周长
		$$p=\frac{a+b+c}{2}$$
	\item 面积
		$$S=\frac{a \cdot H_a}{2}=\frac{ab \cdot sinC}{2}=\sqrt{p(p-a)(p-b)(p-c)}$$
	\item 中线
		$$M_a=\frac{\sqrt{2(b^2+c^2)-a^2}}{2}=\frac{\sqrt{b^2+c^2+2bc \cdot cosA}}{2}$$
	\item 角平分线 
		$$T_a=\frac{\sqrt{bc \cdot [(b+c)^2-a^2]}}{b+c}=\frac{2bc}{b+c}cos\frac{A}{2}$$
	\item 高线
		$$H_a=bsinC=csinB=\sqrt{b^2-(\frac{a^2+b^2-c^2}{2a})^2}$$
	\item 内切圆半径
		\begin{align*}
			r&=\frac{S}{p}=\frac{arcsin\frac{B}{2} \cdot sin\frac{C}{2}}{sin\frac{B+C}{2}}=4R \cdot sin\frac{A}{2}sin\frac{B}{2}sin\frac{C}{2}\\
			&=\sqrt{\frac{(p-a)(p-b)(p-c)}{p}}=p \cdot tan\frac{A}{2}tan\frac{B}{2}tan\frac{C}{2}
		\end{align*}
	\item 外接圆半径
		$$R=\frac{abc}{4S}=\frac{a}{2sinA}=\frac{b}{2sinB}=\frac{c}{2sinC}$$
\end{enumerate}

\subsubsection{四边形}

\subsubsection{正$n$边形}

\subsubsection{圆}

\subsubsection{棱柱}

\subsubsection{棱锥}

\subsubsection{棱台}

\subsubsection{圆柱}

\subsubsection{圆锥}

\subsubsection{圆台}

\subsection{常用数表}

\subsubsection{梅森数}

\end{document}
